\chapter{Úvod}
Webové portály \blindtext

Pro toto téma jsem se rozhodl, protože mám k webovým technologiím blízko již od dětství a chci v tomto směru nadále rozvíjet své znalosti.

Cílem je \blindtext

Tato práce vychází z bakalářské práce \emph{Automaticky aktualizovaný webový portál} Petra Staňka~\cite{bib:stanek}.
Webový portál byl od základu implementován znovu s použitím modernějších technologií. Extrakce dat zůstala zachována, pouze byla rozšířena o extrakci témat výzev k předkládání návrhů.

V druhé kapitole nazvané \emph{Rozbor řešené problematiky} nejprve stručně shrnu základní znaky webových portálu a uvedu několik příkladů stávajících portálů, dále se budu věnovat základním principům vývoje webových stránek obecně. Dále přiblížím několik technologií, kterých jsem při tvorbě této bakalářské práce využil.
Ve třetí kapitole s názvem \emph{Návrh a implementace systému} se již zaměřím na konkrétní návrh a řešení dané problematiky, uvedu několik obrázků z výsledného portálu, jež je výsledkem této práce a popíšu několik zajímavých částí samotného kódu.
Ve čtvrté kapitole nesoucí označení \emph{Experimenty a vyhodnocení} TODO. V samém závěru pak stručně zhodnotím výsledky této práce.