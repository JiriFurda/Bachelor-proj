\chapter{Dosavadní stav}

\section{Webové portály}
Webový portál je druh webové stránky, která shromažďuje informace z vícero různých zdrojů a uživateli ty nejrelevantnější informace prezentuje uživateli shromážděné na jednom místě %[https://www.liferay.com/resources/l/web-portal 20. 4.]. 
Zpravidla je umožněno na portálu v těchto informací vyhledávat. Velmi časté je také zakomponování autentizace uživatele a podle jeho role jsou mu zpřístupněny různé části daného portálu. %[https://kb.iu.edu/d/ajbd]
\blindtext[2]

\subsection{Cordis}
%[https://cordis.europa.eu/about/]
The Community Research and Development Information Service je portál provozovaný Evropskou Unií (dále jen EU) sloužící jako hlavní zdroj výsledků projektů, sponzorovaných v rámci programů EU pro výzkum a inovaci. Na jednom místě veřejně poskytuje informace jak o těchto projektech, tak i o jejích účastnících, o hlášeních, vědeckých zprávách a publikacích.
\blindtext

\subsection{Funding and Tenders}
\blindtext

\section{Flask}
\blindtext[2]

\section{Vue}
\blindtext[2]

\section{Elasticsearch}
Elasticsearch je volně šiřitelný vyhledávač umožňující rychlé vyhledávání ve velkém množství dat. 

Oproti relačním databázím nenabízí propojovací dotazy, a proto je nutné ukládaná data denormalizovat. Díky tomu jsou ale data a jejich metadata v těsné blízkosti a tím je docíleno rychlé fulltextové vyhledávání. %[https://qbox.io/blog/what-is-elasticsearch 20.4.]

\subsection{Dotazy}
Pro komunikaci s touto databází se využívá Query DSL - dotazy aplikačního rozhraní REST ve formátu JSON.

Obsah dotazu může vypadat například následovně:
\begin{verbatim}
{
    "query": {
        "match": {
            "address": "mill lane"
        }
    }
}
\end{verbatim}

Na tento dotaz můžeme od serveru dostat tuto odpověď
\begin{verbatim}
\blindtext
\end{verbatim} 

\subsubsection*{Kontext}
Dotazy v Elasticsearch se rozlišují na dva základní typy - kontext dotazu (query) a kontext filtru.
První z nich se dá lidsky přeložit do otázky \uv{Jak moc dokument splňuje dotaz?}. V tomto případě se u každého výsledku počítá skóre relevance, podle kterého jsou ve výsledné odpovědi dokumenty sestupně řazeny.


\subsection{Základní struktura}
Základní struktura databáze Elasticsearch se na první pohled od tradičních relačních databázi velmi neliší, ke spoustě pojmů se dá najít ekvivalent.

\subsubsection*{Uzel a shluk}
Pod pojmem uzel (anglicky node) se skrývá server, ve kterém jsou uložena data. Uzel v rámci databáze může být jeden nebo jich může být i více a každý bude obsahovat pouze určitou část dat. Tyto uzly jsou seskupeny do shluku (anglicky cluster).


\subsubsection*{Dokument}
Dokument je základní jednotka dat, které může být zaindexována. Obsah dokumentu je zapsán ve formátu JSON. V relační databázi lze dokument přirovnat k řádku tabulky.

\subsubsection*{Index}
Skupina dokumentů s podobnou strukturou se nazývá index. Ekvivalentem v relačních databázich by v tomto případě byla tabulka.

\subsubsection*{Střepy}
Elasticsearch nabízí možnost index rozdělit na střepy (anglicky shards), které mohou být uloženy v různých uzlech. Tuto možnost je vhodné využít především při práci s indexy s velkým obsahem dat. Díky rozdělení docílíme distribuování zátěže a lze využít paralelní zpracování, tudíž bude zpracování dotazu odbaveno výrazně rychleji. 
Přístup k takto rozděleným indexům se nijak nemění a pro uživatele je tento proces naprosto transparentní.

\subsubsection*{Repliky}
Obdobně lze využít i repliky. Zde se ovšem data indexu nerozdělují, ale naopak duplikují. Díky uložení těchto replik mezi více uzly může systém být robustnější a dokáže fungovat i v případě selhání části shluku. I v tomto případě je umožněno vyhledávat paralelně. 

\subsection{Fasetové vyhledávání}
S velkým množstvím dat přichází potřeba obsah podle určitých kriterií zúžit. K tomotu účelu slouží filtry a fasetové vyhledávání (někdy také označována jako fastová navigace - anglicky faceted search nebo faceted navigation). Využití faset je ale pokročilejší než použít filtry, protože fasetová navigace umožňuje využít hned několik filtrů najednou. %[https://www.nngroup.com/articles/filters-vs-facets/ 21.4.]

V Elasticsearch je možné tuto navigaci vytvořit pomocí tzv. kyblíkových agreagací (angl. bucket aggregations).
Jednoduše řečeno, se pro dokumenty vytvoří kyblíky, kde každý z kyblíku odpovídá určitému kriteriu. Pokud dokument toto kriterium splňuje, je do kyblíku vložen.
%[https://www.elastic.co/guide/en/elasticsearch/reference/current/search-aggregations-bucket.html 21.4.]

\subsection{Plnotextové vyhledávání}
\blindtext[2]



\section{Vývoj webu}
\subsection{MVC architektura}
\blindtext[2]

\subsection{Responsivita}
\blindtext[2]