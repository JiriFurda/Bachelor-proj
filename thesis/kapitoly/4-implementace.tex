\chapter{Implementace}

\section{Portál}

\subsection{Fasetové vyhledávání}
Některé s využívaných faset obsahují i stovky položek a jejich hodnoty mají často dlouhé názvy. Jejich zobrazení v postranním panelu jsem považoval za nedostatečné, proto jsem se rozhodl toto řešení oproti předchozímu portálu přepracovat s využitím technologie Vue. Pro každou z faset jsem vytvořil dvě komponenty - položku v postranním panelu a modální okno.

\subsection{Vue komponenty}

\subsubsection{Postranní panel}
Tento panel je tvořen Vue komponentou sidebar-facet-list, která pro každý faset vytvoří instanci komponenty sidebar-facet
Ve výchozím stavu tento panel nabízí nejvýše pět nejčastěji vyskytujících se hodnot daného fasetu. Ty jsou postupně nahrazovány zvolenými hodnotami. Další hodnoty s menším výskytem jsou dostupné v modálním okně.

\subsubsection*{Modální okno}
Okna jednotlivých faset jsou tvořena instancemi Vue komponenty modal-facet. O jejich vytvoření se stará komponenta modal-facet-list
Rozsáhlejší výpis možností daného fasetu se nyní zobrazují v modálním okně. Zde je možnost současně vidět podstatně více možností a také je možné vypsat jejich hodnoty v plném znění. V tomto okně je obsaženo vstupní pole, které asynchronními dotazy na Elasticsearch vyhledává mezi možnostmi daného fasetu.

\subsubsection*{Fasetová možnost}
Jednotlivé fasetové možnosti jsou reprezentovány komponentou option-facet, která zobrazuje zaškrtávací pole, název dané možnosti a počet výsledků aktuálního vyhledávání obsahující tuto možnost. Při zaškrtnutí možnosti se její data zkopírují do proměnné checkedOptions příslušného fasetu. Tato komponenta se vyskytuje jak v modálním okně, tak i v postranním panelu.


\subsubsection*{Inicializace dat}
Na každé stránce portálu se vyskytuje právě jednou komponenta init-facet-data, která jednorázově obstará naplnění proměnných programovacího jazyka JavaScript.

\subsection{Vyhledávání}
Samotné vyhledávání obstarává třída IndexSearch, při svém vytvoření uloží vstupní parametry a sestaví dotazy pro Elasticsearch

\subsubsection*{buildSearch()}
První z dotazů je vytvořen tuto metodou, která zpracuje všechny GET parametry protokolu HTML a v případě shody jejich názvů s některým z faset, přidá tuto hodnotu do dotazu. Tento dotaz se využívá pro hledání v modálním okně fasetu. 

\subsubsection*{buildAggregationsSearch()}
Druhý dotaz vzniká rozšířením dotazu z metody buildSearch(). Je obohacen o agregaci fasetových hodnot, které jsou poté zobrazeny v postranním panelu.

\subsubsection*{prepareLayoutData()}
Poté co byly zmíněné dotazy vykonány je třeba výsledky zpracovat a připravit pro předání do odpovídajících Vue komponent. 