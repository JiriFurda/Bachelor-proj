\chapter{Implementace}

\section{Portál}

\subsection{Struktura adresáře}
Veškeré zdrojové kódy portálu se nachází v adresáři \texttt{portal}. Uvnitř této složky se nachází adresáře odpovídající architektuře MVC.
\begin{itemize}
  \item Složka \texttt{models} obsahuje modely
  \item Složka \texttt{controllers} obsahuje kontrolery
  \item Složka \texttt{templates} obsahuje pohledy
  \item Složka \texttt{static} obsahuje kaskádové styly, zdrojové kódy Vue komponent a využité knihovny
\end{itemize}

\subsection{Vue komponenty}

\subsubsection{Postranní panel}
Tento panel je tvořen Vue komponentou \texttt{sidebar-facet-list}, která pro každý faset vytvoří instanci komponenty \texttt{sidebar-facet}
Ve výchozím stavu tento panel nabízí nejvýše pět nejčastěji vyskytujících se hodnot daného fasetu. Ty jsou postupně nahrazovány zvolenými hodnotami. Další hodnoty s menším výskytem jsou dostupné v modálním okně.

\subsubsection*{Modální okno}
Okna jednotlivých faset jsou tvořena instancemi Vue komponenty \texttt{modal-facet}. O jejich vytvoření se stará komponenta \texttt{modal-facet-list}
Rozsáhlejší výpis možností daného fasetu se nyní zobrazují v modálním okně. Zde je možnost současně vidět podstatně více možností a také je možné vypsat jejich hodnoty v plném znění. V tomto okně je obsaženo vstupní pole, které asynchronními dotazy na Elasticsearch vyhledává mezi možnostmi daného fasetu.

\subsubsection*{Fasetová možnost}
Jednotlivé fasetové možnosti jsou reprezentovány komponentou \texttt{option-facet}, která zobrazuje zaškrtávací pole, název dané možnosti a počet výsledků aktuálního vyhledávání obsahující tuto možnost. Při zaškrtnutí možnosti se její data zkopírují do proměnné \texttt{checkedOptions} příslušného fasetu. Tato komponenta se vyskytuje jak v modálním okně, tak i v postranním panelu.


\subsubsection*{Inicializace dat}
Na každé stránce portálu se vyskytuje právě jednou komponenta \texttt{init-facet-data}, která jednorázově obstará naplnění proměnných programovacího jazyka JavaScript.

\subsection{Vyhledávání}
Samotné vyhledávání obstarává třída \texttt{IndexSearch}, při svém vytvoření uloží vstupní parametry a sestaví dotazy pro Elasticsearch

\subsubsection*{buildSearch()}
První z dotazů je vytvořen tuto metodou, která zpracuje všechny GET parametry protokolu HTML a v případě shody jejich názvů s některým z faset, přidá tuto hodnotu do dotazu. Tento dotaz se využívá pro hledání v modálním okně fasetu. 

\subsubsection*{buildAggregationsSearch()}
Druhý dotaz vzniká rozšířením dotazu z metody \texttt{buildSearch()}. Je obohacen o agregaci fasetových hodnot, které jsou poté zobrazeny v postranním panelu.

\subsubsection*{prepareLayoutData()}
Poté co byly výše zmíněné dotazy vykonány je třeba výsledky zpracovat a připravit pro předání do odpovídajících Vue komponent. Výsledkem této funkce jsou tři struktury JSON, obsahující data využívané pro inicalizaci komponent v šabloně, zabalení ve slovníku.

%TODO: not so interresting
\begin{verbatim}
{
    "facets":[
    {
        "mostFrequentOptions":[  
            {  
               "count":8748,
               "text":"United Kingdom",
               "value":"United Kingdom"
            },
            {  
               "count":5276,
               "text":"Germany",
               "value":"Germany"
            },
            ...
         ],
         "field":"coordinator.country.keyword",
         "name":"coordcountry",
         "checkedOptions":[  
            {  
               "count":4692,
               "text":"Spain",
               "value":"Spain"
            },
         ],
         "title":"Coord. Country"
     },
     ...
     ]
}
\end{verbatim}

\subsection{Vyhledávání hodnot faset}
Protože ve fasetách lze asynchronně vyhledávat, bylo nutné napsat aplikační rozhraní propojující databázi Elasticsearch s Vue komponentou reprezentující toto vyhledávání. Tuto činnost obstarává funkce \texttt{showApi} v kontroleru \texttt{facet\_controller}.

Jako parametr, který je předán cestou URL adresy očekává platný název fasety, pro kterou existuje model třídy \texttt{Facet}. Další parametry jsou předávány metodou GET protokolu HTTP. Prvním povinným parametrem je znění původního dotazu na Elasticsearch, díky tomuto parametru výsledky funkce korespondují s kontextem aktuálního vyhledávání v portálu. Dalším parametrem je pak klíčové slovo nebo jeho část, které se hledá v dostupných hodnotách fasety. 

Po úspěšném dotazu funkce vrací strukturu JSON ve stejném tvaru jako například v \texttt{checkedOptions} zmiňovaném v kapitole Vyhledávání %TODO: reference

\subsection{Model fasety}